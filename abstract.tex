

%----------------------------------------------------------------------------------------
%	PACKAGES AND OTHER DOCUMENT CONFIGURATIONS
%----------------------------------------------------------------------------------------

\documentclass[paper=a4, fontsize=11pt]{scrartcl} % A4 paper and 11pt font size
\usepackage{caption}
\usepackage[T1]{fontenc} % Use 8-bit encoding that has 256 glyphs
\usepackage[english]{babel} % English language/hyphenation
\usepackage{amsmath,amsfonts,amsthm} % Math packages
\usepackage[pdftex]{graphicx}
\usepackage{hyperref}
\usepackage[a4paper]{geometry}

\usepackage{lipsum} % Used for inserting dummy 'Lorem ipsum' text into the template

\usepackage{sectsty} % Allows customizing section commands
\allsectionsfont{\centering \normalfont\scshape} % Make all sections centered, the default font and small caps

\usepackage{fancyhdr} % Custom headers and footers
\pagestyle{fancyplain} % Makes all pages in the document conform to the custom headers and footers
\fancyhead{} % No page header - if you want one, create it in the same way as the footers below
\fancyfoot[L]{} % Empty left footer
\fancyfoot[C]{} % Empty center footer
\fancyfoot[R]{\thepage} % Page numbering for right footer
\renewcommand{\headrulewidth}{0pt} % Remove header underlines
\renewcommand{\footrulewidth}{0pt} % Remove footer underlines
\setlength{\headheight}{13.6pt} % Customize the height of the header

\numberwithin{equation}{section} % Number equations within sections (i.e. 1.1, 1.2, 2.1, 2.2 instead of 1, 2, 3, 4)
\numberwithin{figure}{section} % Number figures within sections (i.e. 1.1, 1.2, 2.1, 2.2 instead of 1, 2, 3, 4)
\numberwithin{table}{section} % Number tables within sections (i.e. 1.1, 1.2, 2.1, 2.2 instead of 1, 2, 3, 4)

\setlength\parindent{10pt} % Removes all indentation from paragraphs - comment this line for an assignment with lots of text

%----------------------------------------------------------------------------------------
%	TITLE SECTION
%----------------------------------------------------------------------------------------

\newcommand{\horrule}[1]{\rule{\linewidth}{#1}} % Create horizontal rule command with 1 argument of height

\newcommand\invisiblesection[1]{%
  \refstepcounter{section}%
  \addcontentsline{toc}{section}{\protect\numberline{\thesection}#1}%
  \sectionmark{#1}}

\title{
\normalfont \normalsize
\huge Technology-based aids for people affected by Autism Spectrum Disorder (ASD)  \\ % The assignment title
}

\author{Davide Peron, Cristina Gava}
\date{\today} % Today's date or a custom date

\begin{document}

\maketitle % Print the title

Individuals affected by Autism Spectrum Disorder(ASD), especially the not-verbal ones, are often unable to communicate in an
appropriate way, they show strong difficulties in social interactions and in manifesting their affective
states or their necessities.
The conventional techniques, used to improve the performances of these people in the
everyday tasks, are observation-based and can require a lot of effort in terms of time and money, with limited
results.
Technology-assisted therapies can result more powerful and fast.

Our aim is to analyze the current technology-based solutions that can help the therapist and the family of an autistic subject to interact with him.
These solutions exploit the joint use of human intelligence and artificial intelligence to improve
the powerfulness of therapies and to allow a better integration of these individuals in the society.
Examples of these type of aids are Virtual Assistants and Agents as therapeutic tools,
wearable technologies or VR headset to help them in everyday communication and to improve their
fundamental and social skills.

Interesting example fields that can be analyzed are:
\begin{itemize}
\item the use of robotic avatar instead of a therapist in order to improve the response and interaction time of the disabled user;
\item a wearable IoT device to collect data about interaction among children with ASD in classroom settings;
\item an autonomous computer system for training social orientation skills to young children with ASD.
\end{itemize}

\end{document}
