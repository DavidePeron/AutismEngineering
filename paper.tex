% \documentclass[journal]{IEEEtran}
\documentclass[12pt,journal,draftclsnofoot,onecolumn]{IEEEtran}
\IEEEoverridecommandlockouts

% remove colon after subsubsection, use spacing instead
\makeatletter
\renewcommand{\@IEEEsectpunct}{\quad}
\makeatother

\makeatletter
% add a small spacing *before* subsections
\let\origsubsubsection\subsubsection
\renewcommand\subsubsection{\@ifstar{\starsubsubsection}{\nostarsubsubsection}}

\newcommand\nostarsubsubsection[1]
{\subsubsectionprelude\origsubsubsection{#1}}

\newcommand\subsubsectionprelude{%
  \vspace{6pt}
}
\makeatother

% change default font to charter
% \usepackage{charter}
% \usepackage[bitstream-charter]{mathdesign}
% \usepackage{XCharter}
\usepackage{amsfonts}

% Fixing IEEEtran.cls bug with [english]{babel}
\makeatletter
\def\markboth#1#2{\def\leftmark{\@IEEEcompsoconly{\sffamily}\MakeUppercase{\protect#1}}%
\def\rightmark{\@IEEEcompsoconly{\sffamily}\MakeUppercase{\protect#2}}}
\makeatother

% \usepackage{t1enc}

\usepackage{listings}
\usepackage{multirow}
%\usepackage[utf8x]{inputenc}
\usepackage[english]{babel}
\selectlanguage{english}
\usepackage{color}
%\usepackage{caption}
\usepackage{cite}
\usepackage[pdftex]{graphicx}

% \usepackage{subfig}
\usepackage{subcaption}
\usepackage{amsmath}

\usepackage{mathtools}
\DeclarePairedDelimiter\ceil{\lceil}{\rceil}
\DeclarePairedDelimiter\floor{\lfloor}{\rfloor}

\usepackage{amsfonts}
\usepackage{array}
\usepackage{verbatim}
\usepackage{listings}
\usepackage{hyperref}
\usepackage{url}
\usepackage{enumerate}
\usepackage{multirow}

\usepackage{siunitx}
\usepackage{epsfig}
\usepackage{epstopdf}
\usepackage{multicol}% http://ctan.org/pkg/multicols
\usepackage[font=footnotesize]{caption}
% \usepackage[font=scriptsize]{subcaption}
% Tikz
\usepackage{tikz}
\usepackage{pgfplots}
\pgfplotsset{compat=newest}
\pgfplotsset{plot coordinates/math parser=false}
\newlength\fheight
\newlength\fwidth
\usetikzlibrary{patterns,decorations.pathreplacing,backgrounds,calc}
\definecolor{SchoolColor}{RGB}{0.71, 0, 0.106}%181,0,27} unipd red
\definecolor{chaptergrey}{rgb}{0.61, 0, 0.09} % dialed back a little
\definecolor{midgrey}{rgb}{0.4, 0.4, 0.4}
\definecolor{chaptergreen}{rgb}{0.09, 0.612, 0}
\definecolor{chapterpurple}{rgb}{0.522, 0, 0.612}
\definecolor{chapterlightgreen}{rgb}{0, 0.612, 0.522}

%\raggedbottom

% Pseudocode
\usepackage[ruled, vlined]{algorithm2e}

\SetKwRepeat{Do}{do}{while}
\SetKwBlock{wpPa}{with probability $P_A$}{end}
\DontPrintSemicolon
% \usepackage{algorithm}
% \usepackage[noend]{algpseudocode}
% \renewcommand\algorithmicthen{}
% \renewcommand\algorithmicdo{}
\usepackage{lscape}

\addto\captionsenglish{\renewcommand{\figurename}{Fig.}}

\newcommand{\field}[1]{\mathbb{#1}}

\DeclareMathOperator*{\argmin}{arg\,min}
\DeclareMathOperator*{\argmax}{arg\,max}
\newcommand{\norm}[1]{\left\lVert#1\right\rVert}
\renewcommand{\arraystretch}{2}

\newcommand{\DP}[1]{\textcolor{blue}{\textbf{(DP says: #1)}}}
\newcommand{\cri}[1]{\textcolor{violet}{\textbf{(Cri says: #1)}}}

\usepackage{threeparttable}
%\usepackage[table,xcdraw]{xcolor}
\usepackage{tabularx}
\usepackage{multirow}
\usepackage{booktabs}
\newcommand{\tabitem}{~~\llap{\textbullet}~~}
\usepackage{array, blindtext}
\usepackage{wrapfig}
\usepackage{pdfpages}
\usepackage[acronym]{glossaries}

% use tikArchiviz images or eps
\newif\iftikz
\tikztrue

\graphicspath{{./figures/}}

\title{Technology-based aids for people affected by Autism Spectrum Disorder (ASD)}
\author{\IEEEauthorblockN{Cristina Gava, Peron Davide}\\
\small{{Department of Information Engineering, University of Padova -- Via Gradenigo, 6/b, 35131 Padova, Italy\\
Email: {\tt\{gavacris, perondav\}@dei.unipd.it}\\}}
}

% Reduce the space below figs.
%\setlength{\belowcaptionskip}{-0.7cm}

\renewcommand{\equationautorefname}{Eq.}

%% Glossary
\newacronym{asd}{ASD}{Autism Spectrum Disorder}

\glsresetall
\begin{document}

\def\equationautorefname~#1\null{(#1)\null}
\setlength{\belowcaptionskip}{-0.2cm}

% reduce space after title
\makeatletter
\patchcmd{\@maketitle}
  {\addvspace{0.5\baselineskip}\egroup}
  {\addvspace{-1.2\baselineskip}\egroup}
  {}
  {}
\makeatother

\maketitle

\begin{abstract}
Individuals affected by Autism Spectrum Disorder(ASD) are often unable to communicate in an
appropriate way, they show strong difficulties in social interactions and in manifesting their affective
states.
The conventional techniques, used to improve the performances of these people in the
everyday tasks, are observation-based and can require a lot of effort in terms of time and money, with limited
results.
Technology-assisted teraphies can result more powerful and fast.

Our aim is to analyze the current technology-based solutions that can help therapists in the treatment
of people affected by ASD.
These solutions exploit the joint use of human intelligence and artificial intelligence to improve
the powerfulness of therapies and to allow a better integration of these individuals in the society.
Examples of these type of aids are Virtual Assistants and Agents as therapeutic tools,
wearable technologies or VR headset to help them in everyday communication and to improve their
fundamental and social skills.
An interesting field that can be analyzed is the use of robotic avatar instead of a therapist, in order to increase
the usefulness of a therapy, improving the response and the interaction time of the disabled user.
\end{abstract}

\glsresetall
\section{Introduction} \label{sec:introduction}

% TODO Brief presentation of ASD (?)
\gls{asd} is a term used to cover a very big set of disorders. In this field there are a lot of studies and experiments, but still is one of the most unknown \DP{forse unknown non è il massimo come termine qui} disease. The symptoms are well known, while the causes are mostly unknown. Nowadays, \textit{screening tests} are used to classify a person as affected by \gls{asd}, but these tests have an high percentage of error (false positive or negative) and they can be administered on a patient at least 3 years old. For this reason, several techniques have been studied to wonder if a children has this disorder in the first 2 years of life.
% TODO Main technologies available until now

\section{Technical Approach}
\label{sec:tech_approach}

\subsection{Simulated Annealing}

\subsection{Jumping Ball}
\subsection{Genetic Algorithm}

\section{Conclusions And Future Work}\label{sec:conclusions}

\bibliographystyle{IEEEtran}
\bibliography{bibliography}

\end{document}

%%% Local Variables:
%%% mode: latex
%%% TeX-master: t
%%% End:
